\section{Подготовка}

Для начала создадим себе рабочий каталог, все наши действия будем производить в
нем:
\begin{lstlisting}
mkdir -p ws/rpi
cd ws/rpi
\end{lstlisting}

Далее нам необходимо убедиться, что Raspberry PI работает, что карта памяти
читается и на карте памяти есть все необходимое для запуска ОС\footnote{ОС это
не только ядро, но и набор утилит нужных для инициализации и работы.}.

Если у вас нет готовой карты памяти с установленным Linux, то создать ее не
трудно. Для начала нужно скачать один из образов доступных по ссылке
\url{https://www.raspberrypi.org/downloads/raspbian/}. Например, вот так я могу
скачать образ Raspbian Jessie Lite от 18 Марта 2016 года:
\begin{lstlisting}
wget http://director.downloads.raspberrypi.org/raspbian_lite/images/raspbian_lite-2016-03-18/2016-03-18-raspbian-jessie-lite.zip
\end{lstlisting}

Образ нужно распкаовать:
\begin{lstlisting}
unzip 2016-03-18-raspbian-jessie-lite.zip
\end{lstlisting}

На выходе вы должны получить файл с расширением \emph{.img}. Теперь нужно
записать этот файл на карту памяти. Допустим, что карта памяти в системе
известна под именем \emph{/dev/mmcblk0}, тогда мы должны сделать следующее:

\begin{lstlisting}
sudo dd if=2016-03-18-raspbian-jessie-lite.img of=/dev/mmcblk0 bs=1M
\end{lstlisting}

Копирование образа на карту памяти может занять некоторое время, так как карты
памяти довольно медленные устройства - это нормально.

После этого вставьте карту памяти в Raspberry PI, подключите питание и
проверьте, что устройство запускается. Имя пользователя по-умолчанию \emph{pi},
а пароль \emph{raspberry}.
