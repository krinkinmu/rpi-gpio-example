\section{Сборка и установка upstream ядра}

Теперь мы хотим поставить на наш Raspberry PI upstream\footnote{Так же
известное как ванильное ядро.} ядро. Вместе с ядром мы также поставим и новый
загрузчик - u-boot.

Для начала нам нужно скачать все нужные части. Начнем с компилятора и
сопутствующих утилит\footnote{То что называется toolchain.}. Репозиторий с
набором утилит располагается на github.com оттуда и возьмем:
\begin{lstlisting}
git clone git://github.com/raspberrypi/tools.git --depth=1
\end{lstlisting}\footnote{Флаг \-\-depth=1 позволяет выкачать только последнюю
версию и не качать всю историю изменений.}

Собирать утилиты не нужно, в репозитории находятся сразу готовые бинарные файлы.

Далее нам нужно выкачать исходники загрузчика:
\begin{lstlisting}
git clone git://git.denx.de/u-boot.git --depth=1
\end{lstlisting}

Конечно, нам понядобятся исходники ядра Linux:
\begin{lstlisting}
git clone git://git.kernel.org/pub/scm/linux/kernel/git/torvalds/linux.git --depth=1
\end{lstlisting}

Наконец нам понадобятся части прошивки Raspberry PI, которые запустят загрузчик,
который, в свою очередь, запустит ядро Linux\footnote{В Raspberry PI первым
запускается "виде процессор" (Video Core), на котором работает своя
проприетарная ОС, и уже Video Core запускает все остальное, включая наш
загрузчик.}:
\begin{lstlisting}
git clone git@github.com:raspberrypi/firmware.git --depth=1
\end{lstlisting}

\subsection{Сборка}

Перед сборкой нам нужно немного настроить наше окружение, а именно нам нужно
экспортировать некоторые переменные окружения, которые нужны для сборки
исходных кодов:
\begin{lstlisting}
export KERNEL_SRC=`pwd`/linux
export CCPREFIX=`pwd`/tools/arm-bcm2708/arm-bcm2708hardfp-linux-gnueabi/bin/arm-bcm2708hardfp-linux-gnueabi-
export ARCH=arm
export CROSS_COMPILE=${CCPREFIX}
\end{lstlisting}

Все это легко можно вынести в скрипт, например, \emph{export.sh}, но чтобы этот
скрипт сделал то, что от него ожидается, его нужно запускать особенным образом
\footnote{В противном случае запуск скрипта породит новый процесс, в котором
будут добавлены переменные окружения, но родительский процесс так и останется
без этих переменных.}:
\begin{lstlisting}
source ./export.sh
\end{lstlisting}

Теперь мы готовы собрать ядро Linux:
\begin{lstlisting}
cd linux
make bcm2835_defconfig
make -j `grep -c ^processor /proc/cpuinfo`
cd ..
\end{lstlisting}

А также наш загрузчик - u-boot:
\begin{lstlisting}
cd u-boot
make rpi_config
make -j `grep -c ^processor /proc/cpuinfo`
cd ..
\end{lstlisting}

Загрузчику нужен конфигурационный файл, который описывает что и как нужно
загрузить. Чтобы подготовить такой файл перейдем в каталог загрузчика:
\begin{lstlisting}
cd u-boot/tools
\end{lstlisting}

Теперь нам нужно, собственно, создать конфигурационный файл для нашего
загрузчика - назовем его \emph{boot.scr}. Его содержимое может быть непонятно,
но очень вряд ли вам в ближайшее время придется что-то в нем менять, поэтому это
не очень страшно - просто будьте внимательны при переписывании:
\begin{lstlisting}
mmc dev 0
setenv fdtfile bcm2835-rpi-b.dtb
setenv bootargs earlyprintk console=tty0 console=ttyAMA0 root=/dev/mmcblk0p2 rootwait
fatload mmc 0:1 ${kernel_addr_r} zImage
fatload mmc 0:1 ${fdt_addr_r} ${fdtfile}
bootz ${kernel_addr_r} - ${fdt_addr_r}
\end{lstlisting}

Если кратко, то этот файл описывает:
\begin{itemize}
  \item где взять образ ядра Linux;
  \item где взять Device Tree Blob (к этому мы еще вернемся);
  \item куда все это нужно положить в памяти;
  \item какие параметры передать ядру Linux при загрузке;
\end{itemize}

Ну и наконец, этому файлу требуется еще кое-какая обработка\footnote{Это особенности
u-boot, которые нас не особо интересуют.}:
\begin{lstlisting}
./mkimage -A arm -O linux -T script -C none -n boot.scr -d boot.scr boot.scr.uimg
\end{lstlisting}

На выходе мы получаем файл \emph{boot.scr.uimg}, его нам нужно будет скопировать
в раздел \emph{boot} нашей карты памяти вместе с ядром Linux и прочими вещами -
мы покажем все это чуть дальше.

\subsection{Установка}

Не забываем вернуться назад в корень нашего рабочего каталога:
\begin{lstlisting}
cd ../..
\end{lstlisting}

Нам нужно записать все что мы подготовили в раздел \emph{boot} нашей карты
памяти. Но предварительно сохраним то, что там было раньше. Допустим что раздел
\emph{boot} примонтирован в системе в каталоге \emph{/media/boot}, тогда нужно
сделать следующее:
\begin{lstlisting}
mkdir /media/boot/backup
mv /media/boot/* /media/boot/backup
\end{lstlisting}

Наконец, теперь мы можем скопировать в \emph{boot} то, что мы хотим там видеть:
\begin{lstlisting}
cp -R firmware/boot/* /media/boot
cp u-boot/u-boot.bin /media/boot/kernel.img
cp linux/arch/arm/boot/zImage /media/boot
cp linux/arch/arm/boot/dts/bcm2835-rpi-b.dtb /media/boot
cp u-boot/tools/boot.scr.uimg /media/boot
\end{lstlisting}

Собственно, скопировали мы в \emph{boot} следующие вещи:
\begin{itemize}
  \item прошивку Raspberry PI (содержимое \emph{firmware/boot});
  \item загрузчик u-boot (в \emph{boot} он называется \emph{kernel.img});
  \item скомпилированное ядрое Linux (\emph{zImage});
  \item Device Tree Blob (\emph{bcm2835-rpi-b.dtb});
  \item конфигурационный файл u-boot (\emph{boot.scr.uimg});
\end{itemize}

На этом мы закончили с установкой upstream ядра Linux на Raspberry PI. Если все
прошло удачно, вы можете загрузить Raspberry PI, но уже с новым ядром.
