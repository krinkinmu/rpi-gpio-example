\section{Device Tree}

Мы не будем углубляться в детали Device Tree (далее просто DT), но чтобы
двигаться дальше нам нужно иметь представление о том, что это такое. Перед тем
как перейти к собственно DT, стоит упомянуть такую вещь как Open Firmware.

Когда ОС загружается ей нужно каким-то образом собрать информацию о системе.
На разных платформах для этого существуют различные интерфейсы: BIOS, EFI, ACPI
и другие. Open Firmware - еще одно такое средство, ранее оно использовалось,
например, на компьютерах построенных на базе PowerPC\footnote{Например, старые
macbook-и.}.

Весь стандарт Open Firware нас не интересует, а интересует только DT. DT, если
просто, это формат, который описывает оборудование системы. Т. е. мы должны
описать оборудование системы в специальном виде и передать его ядру при
загрузке. Ядро ОС при старте прочитает эту информацию и будет знать какое
оборудование присутствует в системе.

\subsection{Raspberry PI Device Tree}

Естественно мы не будем описывать все оборудование, которое есть в Raspberry PI,
за нас уже описали большую часть. Найти описание можно в файле
\emph{linux/arch/arm/boot/dts/bcm2835-rpi-b.dts}.

Как вы видите это обычный текстовый файл с довольно не сложным синтаксисом. При
компиляции ядра этот файл компилируется в бинарный формат, который мы называем
Device Tree Blob. Именно этот бинарный файл вы и скопировали в раздел
\emph{boot} карты памяти.

Файл описывает иерархию "узлов" с набором свойств - пар ключ/значение. Внутри
ядра ОС мы можем обращаться к этому описанию. Назначения свойств зависит от
оборудования, пока нас интересует только свойство \emph{compatible}. Все
узлы со свойством \emph{compatible} описывают некоторое устройство, для которого
нужен драйвер. Собственно, свойство \emph{compatible} позволяет найти драйвер,
который обслуживает данное устройство.

Формат свойства \emph{compatible} очень простой - это просто набор строк
разделенных запятыми. Каждая строка, как правило, состоит из двух частей
разделенных запятой, которые, как правило, описывают производителя оборудования
и модель оборудования. Но в конечном итоге это просто строки, и ядро ОС ищет
подходящий драйвер для устройства просто сравнивая строки\footnote{Детали
остаются на самостоятельное изучение, например, будет полезно разобраться зачем
там нужен набор строк.}.

\subsection{Добавляем свое устройство в Device Tree}

Давайте попробуем модифицировать DT. Пока мы не связаны ни с каким
оборудованием, поэтому пример чисто учебный, но потом мы добавим к описанию
нашего устройства GPIO и получим уже более реалистичный пример.

Найдите в файле \emph{linux/arch/arm/boot/dts/bcm2835-rpi-b.dts} корневой узел,
он начинается с символов \emph{"\/ \{"} и находится в самом начале файла.
Добавьте в конец этого узла новый дочерний узел с именем \emph{example}.
Добавьте в этот узел свойство \emph{compatible}.

У меня новый узел выглядит следующим образом:
\begin{lstlisting}
example {
	compatible = "spbau,example";
};
\end{lstlisting}

Чтобы проверить, что вы все сделали правильно пересоберите ядро\footnote{Это не
должно занять много времени, вы уже собирали его недавно.}. Скопируйте в раздел
\emph{boot} новое ядро и не забудьте про новый Device Tree Blob. Обратите
внимание, вам нужен не файл \emph{linux/arch/arm/boot/dts/bcm2835-rpi-b.dts}, а
его скомпилированная версия - \emph{linux/arch/arm/boot/dts/bcm2835-rpi-b.dtb}.

Загрузите Raspberry PI и далее проверьте каталог \emph{/proc/device-tree}:
\begin{lstlisting}
cd /proc/device-tree
ls
\end{lstlisting}

В выводе команды \emph{ls} вы должны увидеть папку \emph{example}. Если вы ее
видете - значит все сделали правильно.
