\section{Загружаемые модули}

Теперь мы попробуем создать загружаемый модуль для нашего фиктивного устройства,
но перед этим нам необходимо включить поддержку загружаемых модулей в ядре, так
как по-умолчанию для Raspberry PI она выключена:
\begin{lstlisting}
cd linux
ARCH=arm make menuconfig
\end{lstlisting}

Обратите внимание, на то, что перед вызовом \emph{make} мы указываем архитектуру
через переменную окружения \emph{ARCH}. Если вы уже экспортировали нужные
переменные окружения, как было показано ранее, то вам это не нужно.

Далее вам необходимо найти в среди опций конфигурации опцию отвечающую за
загружаемые модули и включить ее. За модули отвечает пункт меню с названием:
\emph{Enable loadable module support}.

Теперь мы можем компилировать наш модуль отдельно от ядра. Впрочем так как мы
используем DT в исходниках ядра, а не собираем его сами выигрыш не велик.

\subsection{Модуль ядра}

Создадим в нашем рабочем каталоге папку, в которой и будут лежать исходные
коды нашего модуля:
\begin{lstlisting}
mkdir example
cd example
\end{lstlisting}

В каталоге создадим, для начала, очень простой загружаемый модуль. В последствии
мы его значительно изменим, но пока нам нужно проверить что мы можем собирать
и загружать модули ядра. Пусть файл называется \emph{example.c}:
\begin{lstlisting}
#include <linux/module.h>
#include <linux/kernel.h>
#include <linux/printk.h>
#include <linux/init.h>

static int __init example_init(void)
{
	pr_info("example: load\n");
	return 0;
}

static void __exit example_exit(void)
{
	pr_info("example: unload\n");
}

module_init(example_init);
module_exit(example_exit);
MODULE_LICENSE("GPL");
\end{lstlisting}

Для сборки модуля нам, конечно, понадобится \emph{Makefile}:
\begin{lstlisting}
ifneq ($(KERNELRELEASE),)

obj-m := example.o

else

KDIR ?= /lib/modules/`uname -r`/build

default:
        $(MAKE) -C $(KDIR) M=$$PWD

clean:
        rm -rf *.o *.ko *.order *.symvers *.mod.c

endif
\end{lstlisting}

Обратите внимание, что путь к исходникам ядра Linux передается в переменной
окружения \emph{KDIR}, и если вы ничего не передадите, то будет использован
каталог, соответствующий запущенной на вашей машине версии ядра в
\emph{/lib/modules}. Это явно не то, что ожидается, так что не забудьте указать
в перменной KDIR путь к правильному каталогу с сконфигурированными исходниками:
\begin{lstlisting}
KDIR=../linux make
\end{lstlisting}

Если вы все сделали правильно, то на выходе у вас должен получиться файл
\emph{example.ko}. Скопируйте его на карту памяти, в любое место, где вам
удобно. Загрузите Rapberry PI и убедитесь, что вы можете загрузит ваш модуль:
\begin{lstlisting}
sudo insmod example.ko
\end{lstlisting}

\subsection{Platform Driver}

Пока наш модуль никаким образом не соотносится с описанием в DT. Для того, чтобы
исправить это нам нужно создать структуру \emph{platform\_driver}, которая будет
описывать наш драйвер.

Если вы это читаете, значит вы уже имеете какое-то представление о ядре Linux.
Например, вы могли слышать или даже писать простые драйвера символьных
устройств. Ядро Linux поддерживаем двольно много различных типов драйверов, но
\emph{platform\_driver} особенный. \emph{platform\_driver}, обычно, описывает
драйвер устройства, которое не может быть автоматически задетектировано. Что
это значит? Обычно устройства подключаются к различным шинам (PCI, USB и др.).
Некоторые шины позволят определять, если к ним было подключено какое-то
устройство и даже могут предоставить какую-то базовую информацию об устройстве:
тип устройства, код производителя и тому подобные вещи. Так вот для устройств
подключаемых к таким шинам мы не будем использовать \emph{platform\_driver}.

Наше фиктивное устройство, очевидно, не подключается ни к какой шине и не может
быть автоматически обнаружено драйвером этой шины, поэтому нам необходимо
использовать \emph{platform\_driver}. Более того, ядро Linux разбирая Device Tree
Blob автоматически для всех узлов, в которых присутствует свойство
\emph{compatible} создаст экземпляр структуры \emph{platform\_device}, который
будет соответствовать узлу DT и передаст его правильному драйверу. Правильный
драйвер, очевидно, ищется по значению свойства \emph{compatible}.

Возвращаясь к коду, после всех изменений он будет выглядеть следующим образом:
\begin{lstlisting}
#include <linux/module.h>
#include <linux/kernel.h>
#include <linux/printk.h>
#include <linux/init.h>

#include <linux/platform_device.h>
#include <linux/of.h>

static int example_probe(struct platform_device *pdev)
{
	pr_info("example: probe\n");
	return 0;
}

static int example_remove(struct platform_device *pdev)
{
	pr_info("example: remove\n");
	return 0;
}

static const struct of_device_id example_of_ids[] = {
	{ .compatible = "spbau,example" },
	{ }
};

static struct platform_driver example = {
	.probe = &example_probe,
	.remove = &example_remove,
	.driver = {
		.name = "example",
		.owner = THIS_MODULE,
		.of_match_table = example_of_ids,
	},
};

MODULE_DEVICE_TABLE(of, example_of_ids);
module_platform_driver(example);
MODULE_LICENSE("GPL");
\end{lstlisting}

На что нужно обратить внимание в этом коде. В первую очередь посмотрите на
массив \emph{example\_of\_ids}\footnote{\emph{of}, в данном случае, это
сокращение от Open Firmware.}. Строка в первом элементе массива должна выглядеть
подозрительно знакомо - это именно та строка, которую мы указали в DT.

Следующий важный момент - структура \emph{example} типа \emph{platform\_driver}.
Мы записываем в нее два указателя на функции: \emph{example\_probe} и
\emph{example\_remove}. Первая будет вызвана когда наш драйвер будет загружен и
ядро найдет устройство, которое обслуживается этим драйвером, а указатель на
структуру \emph{platform\_device} представляющую это устройство будет передан
как параметр.

Вторая функция, как не трудно догадаться, будет вызвана при "удалении"
устройства. В нашем случае это означает одно из двух: либо мы попытались
выгрузить модуль, либо ОС завершает свою работу.

Ну и наконец финальный штрих, вместо двух функций \emph{example\_init} и
\emph{example\_exit} и пары макросов мы теперь используем один макрос
\emph{module\_platform\_driver}, который создаст и зарегистрирует нужные
функции.

Пришло время собрать модуль, и проверить, что новая версия тоже работает как
следует.
